\documentclass[english]{article}
\usepackage[T1]{fontenc}
\usepackage{babel}
\begin{document}

\subsection{Ejercicio 1}

\subsubsection{Introducci\'{o}n}

Uno de los problemas de la era digital que se han enfrentado, es el
hecho de representar n\'{u}meros, puesto que para ello, solo se tiene
una cantidad finita de d\'{\i}gitos. Para esto, se contemplaron dos
solucienes: escribir los numeros en formato punto fijo, o en formato
punto flotante, tanto para n\'{u}meros no signados como signados.
En este ejercicio, hablaremos del punto fijo.

El punto fijo es una forma de representaci\'{o}n num\'{e}rica que
consiste en previamente establecer la cantidad de d\'{\i}gitos que
se desean para la parte entera, y establecer otra cantidad de d\'{\i}gitos
para la parte fraccionaria ( m\'{a}s conocida como la que ``est\'{a}
detras de la coma''). Como los d\'{\i}gitos pueden ser establecidos
de manera arbitraria, existen par\'{a}metros para definirlos, a saber:
resoluci\'{o}n, rango y exactitud. 

La resoluci\'{o}n consiste en determinar la magnitud m\'{a}s peque\~{n}a
que es posible de representar con la cantidad de d\'{\i}gitos elegidos.
El rango referencia la cantidad neta que es posible de representar
con los d\'{\i}gitos elegidos, ya que el valor sale de la resta entre
el n\'{u}mero m\'{a}s grande representable menos el m\'{a}s peque\~{n}o
representable. Por \'{u}ltimo, la exactitud consiste en el m\'{a}ximo
error que se comete entre un n\'{u}mero real y su representaci\'{o}n.

\subsubsection{Objetivo}

El objetivo del presente ejercicio es realizar un programa en el cual,
dandole una cantidad de cifras de la parte entera, la cantidad de
cifras de la parte fraccionaria y la condici\'{o}n de signado o no
signado, devuelva la resoluci\'{o}n y el rango del sistema conformado.

\subsubsection{Funcionamiento del programa}

El programa fue realizado en el lenguaje C. Se le pasan por la l\'{\i}nea
de par\'{a}metros los siguientes valores: signado/no signado, cantidad
de bits de la parte entera, cantidad de bits de la parte fraccionara.
En primera instancia, el programa eval\'{u}a que la cantidad de valores
ingresados sea la correcta. Luego, extrae los valores para poder trabajar
con ellos y verifica que \'{e}stos ingresados sean validos, a saber:
el valor signado/no signado acepta el valor 0 como no signado y 1
o '-' como signado; los otros valores s\'{o}lo es posible ingresar
n\'{u}meros.

Luego, procede a hacer las cuentas para devolver los par\'{a}metros
estipulados, se basa en trabajar con potencias de 2 tanto negativas
como positivas, y seg\'{u}n si el parametro es signado o no signado.Por
\'{u}ltimo, se imprimen en pantalla los valores calculados.

Para finalizar, destacamos que se realiz\'{o} una funci\'{o}n propia para hacer la operaci\'{o}n
$2^{n}$ un poco mas sencilla.


\end{document}
